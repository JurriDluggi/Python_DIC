\subsubsection{Installing using default Python3 (CPython) distribution}
\label{subs:Download and Install CPython}
\indent\indent In order to run Python3 on your computer, a Python environment needs to be installed. The default Python distribution is CPython maintaned by Python Software Foundation. \\
\newline
\indent Visit \textit{\href{https://www.python.org/downloads/}} and download the latest stable Python3 distribution \underline{3} (Python 3.x 64 bits) package for Windows. Follow then the installation instructions, allow installation of PIP package manager and adding Python to the system path.

\subsubsection{Installing using Miniconda3}
\label{subs:Installing using Miniconda3}
\indent\indent In order to run Python on your computer, a Python environment needs to be installed. Using a package manager as Miniconda (a light version of Anaconda) simplify the process and provides a good management tool for external libraries.\\
\newline
\indent Visit \textit{\href{http://conda.pydata.org/miniconda.html}} and download the last Miniconda \underline{3} (Python 3.x 64 bits) package for Windows. Follow then the installation instructions.

\subsubsection{Install Required Libraries}
\label{subs:Install Required Libraries}
\indent\indent Several libraries are compulsory for the PythonDIC software to run properly. Some of the libraries are distributed as pre-compiled binary wheels. Many of them can be also found here: \textit{\href{http://www.lfd.uci.edu/~gohlke/pythonlibs/}} .

The installation process using the PIP package manager is simple and convenient. Here is how it works:
\begin{itemize}
  \item Open the Command Window (\texit{Start \textgreater \space Search \textgreater \space cmd})
  \item Type in : \textit{pip install matplotlib}, or \textit{conda install matplotlib}
  \item Follow the instruction to install the package
  \item Do the same with the \textit{scipy} library
\end{itemize}
\newline
\indent\indent OpenCV3 is a library which also needs to be installed. \\
\newline
\indent For CPython version of OpenCV3 compiled version for Windows systems is available here: \textit{\href{http://www.lfd.uci.edu/~gohlke/pythonlibs/}}. Here is how to install it.
\begin{itemize}
	\item Download the package wheel pre-compiled for your Python version
  \item Install the wheel using Command : \textit{pip install [file]}
\end{itemize}
\newline
\indent The OpenCV release available on the default conda channel is not compatible with the latest Python3.x version when I'm writing these lines. However, the correct version of OpenCV have been compiled by different people and is available on different channels.\\
\newline
\indent The OpenCV3 compiled version for Windows systems is available on the \textit{menpo} channel. Here is how to install it.
\begin{itemize}
  \item Terminal Command : \textit{conda install -c menpo opencv3}
\end{itemize}
\newline
\indent\indent In case the package is not available on this channel, use \textit{anaconda search opencv3} to find another place to download it from. (You may need the anaconda client to search for libraries. To install it, use \textit{conda install anaconda-client}).\\

\indent Libraries have been installed. The software is ready to be started.

\subsubsection{Start Python DIC}
\label{subs:Start Python DIC}
  \indent\indent Once the Python DIC files have been download from the GitHub repository (and extracted if needed). Follow these simple steps.
  \begin{itemize}
    \item Navigate to the folder where the Python DIC files have been extracted.
    \item Right-click on DIC.py and select \textit{Open With..}
    \item Select \textit{Browse}
    \item Navigate to your Miniconda folder (Default: \textit{C:/Users/\textless SessionName\textgreater /Miniconda3})
    \item Chose \textit{pythonw.exe} as a default program.
  \end{itemize}
  \newline
  \indent\indent The program can also be started using the windows Command Line. Navigate to the Python DIC folder with the \textit{cd} command and start the Python DIC program using : \textit{python DIC.py}.\\
  \newline

