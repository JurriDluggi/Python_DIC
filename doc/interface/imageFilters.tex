\newline
\indent\indent Different filters can be applied to your set of images before any analysis. These filters allow you to improve the quality of the correlation by performing operations on your images such as contrast enhancement or noise reduction.\\
\newline
\indent The Python DIC software will \textbf{not} modify any of your images. Filters are applied after the reading process and do \textbf{not} affect the original files.\\
\newline
Filters are simple and convenient to use:
\begin{itemize}
  \item Select one of the available filters
  \item Modify parameters specific to the selected filter
  \item Click on the \textit{Preview} button to have a instant view of the filter effect
  \item Click the \textit{Apply Filter} button when you're satisfied to add the filter to applied filters list
  \item Select any filter in the applied filter list and click the \textit{Delete Selection} button to remove the filter from the applied filters list
\end{itemize}
\newline
\indent \textit{Note}: An histogram is available and provides an instant view of the gray scale dispersion on the current image.
